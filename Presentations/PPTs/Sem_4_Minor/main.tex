\documentclass{beamer}
\usepackage{graphicx}
\usepackage{hyperref}
\usepackage[sorting=none]{biblatex}
\usepackage[font=small,skip=0pt]{caption}
\bibliography{refs}
\usepackage{tikz}
\renewcommand{\footnotesize}{\fontsize{6pt}{6pt}\selectfont}
\usetheme{Madrid}
\usecolortheme{beaver}
\titlegraphic{\includegraphics[width=2.5cm]{logo.png}}
\title[High Harmonic Generation]{High Harmonic Generation in Laser Plasma Interaction}
\date{}
\institute[IIT Delhi]{\large Indian Institute of Technology, Delhi}
\author[]{Kulwinder Kaur (2021PHS7190)\\ Harikesh Kushwaha (2021PHS7181)\\[3mm]Adviser: Prof. Vikrant Saxena}
\vspace{0cm}
\begin{document}
\maketitle

\begin{frame}{Introduction}
    \frametitle{Introduction}
    \small
    Ultra high light intensity interactions with matter provide an opportunity to investigate new physical phenomena that have yet to be explored or have been only minimally explored in laboratory settings.
    \begin{itemize}
        \item Intensity of $10^{23} \; W/cm^{2}$ has been reached experimentally.\footnote{\textit{Henri Vincenti} 10.1103/physrevlett.123.105001}
        \item QED at $I = 10^{25}W/cm^{2}$. Schwinger field at $I = 10^{29}W/cm^{2}$.\footnote{\textit{Jin Woo Yoon et al} 10.1364/OPTICA.420520}
        \item Plasma is overdense if $\omega<\omega_p$.
        \item Harmonics are generated by interaction of laser with overdense plasma.
    \end{itemize}
    \begin{minipage}[h]{0.48\linewidth}
        \centering
        \includegraphics[width=0.9\textwidth, height=0.42\textheight]{images/field.jpg}
        \label{fig:field}
    \end{minipage}
    \begin{minipage}[h]{0.48\linewidth}
        \begin{figure}
            \centering
            \includegraphics[width=0.9\textwidth, height=0.42\textheight]{images/harmonics.png}
            \label{fig:harmonics}
        \end{figure}
    \end{minipage}
\end{frame}

\begin{frame}
    \frametitle{Summary of Work Done in the Previous Semester}
    \small
    \begin{itemize}
        \item Interaction of highly intense laser pulse with overdense and underdense plasma
        \item Change in effective critical density of plasma for relativistic laser pulse
        \item The oscillations of plasma surface.
              \begin{itemize}
                  \item Oscillations increases with increase in intensity
                  \item Surface oscillations have even harmonics
              \end{itemize}
        \item Study of high harmonics generation in laser plasma interaction (normal incidence)
              \begin{itemize}
                  \item Only odd harmonics are generated
                  \item A resonance at $n_0/n_c=4$ is observed
                  \item Increasing intensity and pulse duration increases number of harmonics
                  \item No effect due to envelopes
              \end{itemize}
    \end{itemize}

    \textbf{What Now?}
    \begin{itemize}
        \item Effect of Super Gaussian (SG) envelopes on the generated high harmonics
        \item Oblique incident and p- and s-polarized laser
        \item Selection rules for p- and s-polarized laser
    \end{itemize}
\end{frame}

\begin{frame}
    \frametitle{Simulation Details}
    \small
    We want to study the effect of super Gaussian envelope on the generated high harmonics. We performed some simulations in 1D3V. Here are some parameters:

    \begin{minipage}[t]{0.48\linewidth}
        \begin{itemize}
            \item Particles per cell: 100
            \item Number of cells: 16000
            \item Wavelength $\lambda_l = 1 \mu m$
            \item Pulse duration $= 20 \tau$ ($\tau\approx 3.3 fs$)
            \item Simulation time $= 40 \tau$
            \item Intensity of laser for $a_0 = 0.5$ is $I = 3.425 \times 10^{17} W/cm^2$
            \item The density to critical density ratio is $n_0/n_c = 4$
        \end{itemize}
        We also performed simulations with p- and s-polarized laser incidence at oblique angle.
    \end{minipage}
    \begin{minipage}[t]{0.48\linewidth}
        \begin{figure}
            \centering
            \includegraphics[width=1.0\textwidth, height=0.62\textheight]{images/plasma.jpg}
            \label{fig:plasma}
        \end{figure}
    \end{minipage}
\end{frame}


\begin{frame}
    \frametitle{Oblique Incidence: Transformations}
    \small
    \begin{itemize}
        \item We follow Bourdier\footnote{\textit{Bourdier,A. } 10 . 1063 / 1.864355} to make a transformation which lets us simulate oblique incidence in 1D.
    \end{itemize}
    \begin{minipage}[t]{0.35\linewidth}
        \begin{align*}
            \begin{split}
                &\text{For p-polarization}\\
                \mathbf{E}_L  & = E_0(-\sin\alpha \hat{x} + \cos\alpha \hat{y}) \\
                \mathbf{E}_M  & = E_0\cos\alpha \hat{y}                         \\
                c\mathbf{B}_L & = E_0\hat{z}                                    \\
                c\mathbf{B}_M & = E_0\cos\alpha \hat{z}\\
                &\text{For s-polarization}\\
                \mathbf{E}_L & = E_0\hat{z}                                    \\
                \mathbf{E}_M & = E_0\cos\alpha \hat{z}                        \\
                c\mathbf{B}_L & = E_0(\sin\alpha \hat{x} - \cos\alpha \hat{y}) \\
                c\mathbf{B}_M & = -E_0\cos\alpha \hat{y}
            \end{split}
        \end{align*}

    \end{minipage}
    \begin{minipage}[t]{0.60\linewidth}
        \begin{figure}
            \centering
            \includegraphics[width=1.0\textwidth, height=0.62\textheight]{images/frames.png}
            \label{fig:frames}
        \end{figure}
    \end{minipage}
\end{frame}

\begin{frame}
    \frametitle{p- and s- Polarized Laser: Selection Rule}
    \begin{minipage}[h]{0.18\linewidth}
        \small{p-Polarization}
        \tiny{
            \textbf{p-Polarized:}Even, Odd

            \textbf{s-Polarized:} None}
    \end{minipage}
    \begin{minipage}[h]{0.8\linewidth}
        \centering
        \includegraphics[width=0.9\textwidth, height=0.42\textheight]{images/p.png}
        \label{fig:p}
    \end{minipage}

    \begin{minipage}[h]{0.8\linewidth}
        \centering
        \includegraphics[width=0.9\textwidth, height=0.42\textheight]{images/s.png}
        \label{fig:s}
    \end{minipage}
    \begin{minipage}[h]{0.18\linewidth}
        \small{s-Polarization}
        \tiny{
            \textbf{p-Polarized:} Even

            \textbf{s-Polarized:} Odd}
    \end{minipage}

\end{frame}

\begin{frame}
    \frametitle{Results: SG Envelope}
    \begin{figure}[h]
        \centering
        \includegraphics[width=0.85\textwidth]{images/sg.png}
        \caption{The spectrum of SG envelopes with power 2,4,6, and 8 is shown in a sigle plot. A small increase in the peak amplitude is observed with increasing power.}
        \label{fig:sg}
    \end{figure}
\end{frame}

\begin{frame}
    \frametitle{Results: p-Polarized Laser}
    \begin{figure}[h]
        \centering
        \includegraphics[width=0.75\textwidth]{images/p_fft.png}
        \caption{\small{Spectrum of HHG for p-polarized light. Simulation parameters are $\alpha = \pi/4$, the density is $n_0 = 7n_c$ and $a_0 = 4$}}
    \end{figure}
\end{frame}

\begin{frame}
    \frametitle{Results: s-Polarized Laser}
    \begin{figure}
        \centering
        \includegraphics[width=0.75\textwidth]{images/s_fft.png}
        \caption{\small{Spectrum of HHG for s-polarized light. Simulation parameters are $\alpha = \pi/4$, the density is $n_0 = 7n_c$ and $a_0 = 4$}}
    \end{figure}
\end{frame}

\begin{frame}
    \frametitle{Current Status and Future Plan of Work}
    \small
    \textbf{Current Status}
    \begin{itemize}
        \item A brief overview of HHG generation in laser plasma interaction
        \item SG envelopes have very little effect on the generated harmonics
        \item For p-polarized laser, even and odd p-polarized harmonics.
        \item For s-polarized laser, odd s-polarized harmonics and even p-polarized harmonics.
    \end{itemize}
    \textbf{Future Plan of Work}
    \begin{itemize}
        \item Study oblique incidence and polarization more rigorously.
        \item Do 2D simulations.
        \item Compare it with the 1D results.
    \end{itemize}
    \textbf{Acknowledgement}
    We would like to extend our sincerest gratitude to Professor Vikrant Saxena for his unwavering support, patience, motivation, enthusiasm, and invaluable guidance.
\end{frame}
\end{document}